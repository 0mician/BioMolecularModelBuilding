\documentclass[11pt, a4paper,titlepage]{article}
\usepackage[utf8]{inputenc}
\usepackage[T1]{fontenc}
\usepackage{fixltx2e}
\usepackage{graphicx}
\usepackage{longtable}
\usepackage{float}
\usepackage{wrapfig}
\usepackage{soul}
\usepackage{textcomp}
\usepackage{marvosym}
\usepackage{wasysym}
\usepackage{latexsym}
\usepackage{hyperref}
\usepackage{amssymb}
\usepackage{hyperref}
\tolerance=1000
\usepackage[left=2.35cm, right=3.35cm, top=3.35cm, bottom=3.35cm]{geometry}
\usepackage[utf8]{inputenc}
\usepackage[greek,english]{babel}
\usepackage{graphicx}
\usepackage{titlesec}
\usepackage{tocbibind}

\begin{document}

\begin{titlepage}
  \begin{center}
    
    \includegraphics[scale=1.5]{Figures/kuleuven_logo.pdf}~\\[4.5cm]
    
    \textsc{\Large Bio-Molecular Model Building}\\[0.5cm]
    
    % Title
    \rule{\linewidth}{0.3mm}\\[0.4cm]
    {\huge \bfseries Exam Exercise} \\[0.4cm]
    {\large Spring 2015} \\[0.4cm]
    \rule{\linewidth}{0.3mm}\\[1.5cm]
    
    % Author and supervisor
    \begin{minipage}{0.4\textwidth}
      \begin{flushleft} \large
        \emph{Author:}\\
        Cedric \textsc{Lood}\\
        Yi Ming \textsc{Gan}\\
      \end{flushleft}
    \end{minipage}
    \begin{minipage}{0.4\textwidth}
      \begin{flushright} \large
        \emph{Supervisors:} \\
        M. \textsc{De Maeyer}\\
        J. \textsc{De Raeymaecker}\\
        X. \textsc{Qing}
      \end{flushright}
    \end{minipage}
    
    \vfill
    
    \includegraphics[scale=0.15]{Figures/KUL.jpg}~\\[0.5cm]

    % Bottom of the page
    {\large \today}
    
  \end{center}
\end{titlepage}

\setcounter{tocdepth}{3}
\tableofcontents
\clearpage


\section{Question 1}
\subsection{Part a}

Chain A is the kinase domain (CDK2), N-terminal is colored in blue and
C-terminal is colored in green.  The regulatory domain (cyclin-A2) is
chain B, colored in yellow.
\includegraphics[width=15cm]{./Figures/1a.jpg}

\subsection{Part b}
This is the list of residue sequences associated with their
corresponding secondary structures::
\begin{description}
\item[1-4] Loop
\item[5-11] Beta sheet
\item[12-16] Loop
\item[17-23] Beta sheet
\item[24-28] Loop
\item[29-36] Beta sheet
\item[37-45] Loop
\item[46-57] Alpha helix
\item[58-65] Loop
\item[66-71] Beta sheet
\item[72-74] Loop
\item[75-81] Beta sheet
\item[82-84] Loop
\end{description}
And the topology diagram of the N-terminal domain:

\includegraphics[width=12cm]{./Figures/1b.png}

\section{Question 2}
\includegraphics[width=15cm]{./Figures/2a.png}

The role of regulatory domains in the kinases are to induce
conformational changes that switch the kinase from one form (inactive
or active) to the other \cite{ConformationalPlasticityKinases}.

\includegraphics[width=15cm]{./Figures/2b.png}

\section{Question 3}

We extracted the data from the protein database using the following query:

\begin{verbatim}
Holdings : Molecule Type=ignore Experimental Method=X-RAY 

and 

DepositDateQuery: database_PDB_rev.date_original.comparator=between 
database_PDB_rev.date_original.min=2000-01-01 database_PDB_rev.date_original.max=2015-05-13 
database_PDB_rev.mod_type.comparator=< database_PDB_rev.mod_type.value=1 

and 

StructTitleQuery: struct.title.comparator=contains struct.title.value=Kinase
\end{verbatim}

We did export the results to a CSV file to further our axsnalysis using
R and obtained the following summary table and associated graph:

\begin{verbatim}
> table(years) years 2000 2001 2002 2003 2004 2005 2006 2007 2008 2009
 2010 2011 2012 2013 2014 2015 41 84 91 131 123 192 224 298 329 309
 348 316 426 402 358 102 > barplot(table(years))
\end{verbatim}

\includegraphics[width=12cm]{./Figures/pdb_kinases_growth.png}

We think that the steady increase in the number of kinases being
researched is linked to the fact that Kinases are involved in various
pathways whose defects lead to diseases. 

\section{Question 4}
\subsection{Part a}

The molecule bound is Phosphoaminophosphonic Acid-Adenylate Ester, or
ANP. Along with 2 Magnesium Ions.

\subsection{Part b}
ANP is an analog of ATP that cannot be hydrolized by the
kinase. Therefore it stays bound to the active site of the kinase and
allows for the crystal structure of the molecule to be established.

\subsection{Part c}

\includegraphics[width=15cm]{./Figures/4c.png}

As shown in the figure, a salt bridge is formed between Glu89 and
Lys67.  Lys67 forms a salt bridge directly with the $\alpha$-phosphate
oxygen of the ANP molecule. Asp186 forms H-Bond with the Lys67 and
also coordinates the Magnesium Ion, that in turns coordinates the
$\beta$-phosphate oxygen of the ANP molecule. Asn172, in collaboration
with Asp186 coordinates the second Magnesium Ion which interacts with
the $\alpha$ and $\gamma$-phosphate oxygens of the ANP molecule
\cite{CSPim1Kinase}.

\subsection{Part d}

The AUTHOR section from the PDB file reveals the same list of names as
the list of the article's authors:

\begin{verbatim}
AUTHOR    K.C.QIAN,L.WANG,E.R.HICKEY,J.STUDTS,K.BARRINGER,C.PENG,               
AUTHOR   2 A.KRONKAITIS,J.LI,A.WHITE,S.MISCHE,B.FARMER         
\end{verbatim}
 
\section{Question 5}
We found 2 proteins of interest: 1JKK and 3F5U. Both have reasonable
resolutions and R-Free values are similar. 1JKK boasts a 'up to 1.5 A'
resolution of the catalytic domain. However, after visualizing the
B-Factors using pymol, we decided to go with 3F5U.

\subsection{Part a}
The AUTHOR section from the PDB file reveals the same list of names as
the list of the article's authors:

\begin{verbatim}
AUTHOR    L.K.MCNAMARA,D.M.WATTERSON,J.S.BRUNZELLE
\end{verbatim}

\subsection{Part b}
Arg156 and Glu107 have zero occupany. Since the orientation of the
residues in space could not be determined, the interaction of zero
occupancy residue with neighbouring residues is unknown. This may in
turn affect the protein model tertiary structure.  Gln223 has multiple
alternative rotamer conformations.

\includegraphics[width=15cm]{./Figures/5b.png}

\subsection{Part c}

Overall, this structure has low B-factors. There is however a loop
region, located around amino acids 291-294, which displays higher
B-Factors (around 100). The reason for this seems to be lying in the
fact that this loop is part of a flexible regions at the C-Terminus,
which is not captured by the X-Ray crystalography.

\includegraphics[width=15cm]{./Figures/5c.png}

\subsection{Part d}
The domains present in DAPK1-Human
\footnote{\url{http://www.uniprot.org/uniprot/P53355}} are:

\begin{itemize}
\item Kinase whose function is to catalize the transfer of phosphate
  groups to specific substrates.
\item Ankyrin domains (multiple found) mediate protein-protein
  interactions.
  \footnote{\url{https://en.wikipedia.org/wiki/Ankyrin_repeat}}
\item Roc domain, which is a GTPase domain. GTP binding to the ROC
  domain activates kinase activity.
  \footnote{\url{http://www.copewithcytokines.de/cope.cgi?key=ROC\%20domain}} 
\item The death domain (DD) is a protein interaction module composed
  of a bundle of six alpha-helices.
  \footnote{\url{https://en.wikipedia.org/wiki/Death_domain}}
\end{itemize}
\subsection{Part e}

The sequences provided are all homologous and display sequence
similarity.Some of the organism in which the protein is found belong
to varied kingdom such as Animal, Bacteria, Plants, which indicates
that it must originates from an ancient gene that existed even before
the split between Eukaryotes and Prokaryotes on the tree of life.

To further our analysis, we used the guide tree created when we
performed the multiple sequence alignment, along with the results of
the alignment. We found that the gene had had multiple duplication
event during evolution. For example, we analyzed the duplication event
(indicated on the figure) that gave rise to the proteins indicated by
LRRK1, LRRK2, and LRRK3. All of which can be fount in organism
belonging to the Eukaryotic domain. On the picture below, you can see
that we emphasized 3 sub-clusters. The green and the red cluster show
the LRRK1 and LRRK2+LRRK3 groups (the blue cluster can be hypothesized
to have come from a duplication event early in the
Animals/Invertebrates branch).

Using the results from the sequence alignment, you can observe that
the alignment scores between the protein within the same cluster (for
example LRRK1) are better than the ones from another cluster (LRRK2 in
our example). This makes sense when you think about the evolutionary
distance between the 2 genes coding for these paralog genes.

\includegraphics[width=17cm]{./Figures/5e.png}

\section{Question 6}

The essential features are indicated in the red frame on the cartoon
below. They are all hydrogen bridges.

\includegraphics[width=17cm]{./Figures/6.pdf}

\section{Question 7}

\bibliographystyle{plain}
\bibliography{bib-db}
\end{document}
