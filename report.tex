\documentclass[11pt, a4paper,titlepage]{article}
\usepackage[utf8]{inputenc}
\usepackage[T1]{fontenc}
\usepackage{fixltx2e}
\usepackage{graphicx}
\usepackage{longtable}
\usepackage{float}
\usepackage{wrapfig}
\usepackage{soul}
\usepackage{textcomp}
\usepackage{marvosym}
\usepackage{wasysym}
\usepackage{latexsym}
\usepackage{amssymb}
\usepackage{hyperref}
\tolerance=1000
\usepackage[left=2.35cm, right=3.35cm, top=3.35cm, bottom=3.35cm]{geometry}
\usepackage[utf8]{inputenc}
\usepackage[english]{babel}
\usepackage{graphicx}
\usepackage{titlesec}
\usepackage{tocbibind}
\providecommand{\alert}[1]{\textbf{#1}}

\begin{document}

\setlength{\parskip}{0pt}%
\setlength{\parindent}{0pt}%
\renewcommand{\thesubsubsection}{\alph{subsubsection}.)}
\begin{titlepage}
  \begin{center}
    
    \includegraphics[scale=1.5]{Figures/kuleuven_logo.pdf}~\\[4.5cm]
    
    \textsc{\Large Bio-Molecular Model Building}\\[0.5cm]
    
    % Title
    \rule{\linewidth}{0.3mm}\\[0.4cm]
    {\huge \bfseries Exam Exercise} \\[0.4cm]
    {\large Spring 2015} \\[0.4cm]
    \rule{\linewidth}{0.3mm}\\[1.5cm]
    
    % Author and supervisor
    \begin{minipage}{0.4\textwidth}
      \begin{flushleft} \large
        \emph{Author:}\\
        Cedric \textsc{Lood}\\
        Yi Ming \textsc{Gan}\\
      \end{flushleft}
    \end{minipage}
    \begin{minipage}{0.4\textwidth}
      \begin{flushright} \large
        \emph{Supervisors:} \\
        M. \textsc{De Maeyer}\\
        J. \textsc{De Raeymaecker}\\
        X. \textsc{Qing}
      \end{flushright}
    \end{minipage}
    
    \vfill
    
    \includegraphics[scale=0.15]{Figures/KUL.jpg}~\\[0.5cm]

    % Bottom of the page
    {\large \today}
    
  \end{center}
\end{titlepage}

\setcounter{tocdepth}{3}
\tableofcontents
\clearpage


\section{Question 1 - Kinases}
\subsection{Part a}
\includegraphics[width=10cm]{./Figures/1a.jpg}

\subsection{Part b}
\includegraphics[width=10cm]{./Figures/1b.pdf}

\section{Question 2 - Kinase active/inactive forms}
\includegraphics[width=10cm]{./Figures/2a.png}

The role of regulatory domains in the kinases are to induce
conformational changes that switch the kinase from one form (inactive
or active) to the other \cite{ConformationalPlasticityKinases}.

\includegraphics[width=10cm]{./Figures/2b.png}

\bibliographystyle{plain}
\bibliography{bib-db}
\end{document}
